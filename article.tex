% Options for packages loaded elsewhere
\PassOptionsToPackage{unicode}{hyperref}
\PassOptionsToPackage{hyphens}{url}
%
\documentclass[
  10pt,
  draftcls,
  technote,
  letterpaper,
  oneside,
  onecolumn]{IEEEtran}
\usepackage{amsmath,amssymb}
\usepackage{iftex}
\ifPDFTeX
  \usepackage[T1]{fontenc}
  \usepackage[utf8]{inputenc}
  \usepackage{textcomp} % provide euro and other symbols
\else % if luatex or xetex
  \usepackage{unicode-math} % this also loads fontspec
  \defaultfontfeatures{Scale=MatchLowercase}
  \defaultfontfeatures[\rmfamily]{Ligatures=TeX,Scale=1}
\fi
\usepackage{lmodern}
\ifPDFTeX\else
  % xetex/luatex font selection
\fi
% Use upquote if available, for straight quotes in verbatim environments
\IfFileExists{upquote.sty}{\usepackage{upquote}}{}
\IfFileExists{microtype.sty}{% use microtype if available
  \usepackage[]{microtype}
  \UseMicrotypeSet[protrusion]{basicmath} % disable protrusion for tt fonts
}{}
\makeatletter
\@ifundefined{KOMAClassName}{% if non-KOMA class
  \IfFileExists{parskip.sty}{%
    \usepackage{parskip}
  }{% else
    \setlength{\parindent}{0pt}
    \setlength{\parskip}{6pt plus 2pt minus 1pt}}
}{% if KOMA class
  \KOMAoptions{parskip=half}}
\makeatother
\usepackage{xcolor}
\usepackage{longtable,booktabs,array}
\usepackage{calc} % for calculating minipage widths
% Correct order of tables after \paragraph or \subparagraph
\usepackage{etoolbox}
\makeatletter
\patchcmd\longtable{\par}{\if@noskipsec\mbox{}\fi\par}{}{}
\makeatother
% Allow footnotes in longtable head/foot
\IfFileExists{footnotehyper.sty}{\usepackage{footnotehyper}}{\usepackage{footnote}}
\makesavenoteenv{longtable}
\usepackage{graphicx}
\makeatletter
\def\maxwidth{\ifdim\Gin@nat@width>\linewidth\linewidth\else\Gin@nat@width\fi}
\def\maxheight{\ifdim\Gin@nat@height>\textheight\textheight\else\Gin@nat@height\fi}
\makeatother
% Scale images if necessary, so that they will not overflow the page
% margins by default, and it is still possible to overwrite the defaults
% using explicit options in \includegraphics[width, height, ...]{}
\setkeys{Gin}{width=\maxwidth,height=\maxheight,keepaspectratio}
% Set default figure placement to htbp
\makeatletter
\def\fps@figure{htbp}
\makeatother
\setlength{\emergencystretch}{3em} % prevent overfull lines
\providecommand{\tightlist}{%
  \setlength{\itemsep}{0pt}\setlength{\parskip}{0pt}}
\setcounter{secnumdepth}{5}
\newlength{\cslhangindent}
\setlength{\cslhangindent}{1.5em}
\newlength{\csllabelwidth}
\setlength{\csllabelwidth}{3em}
\newlength{\cslentryspacingunit} % times entry-spacing
\setlength{\cslentryspacingunit}{\parskip}
\newenvironment{CSLReferences}[2] % #1 hanging-ident, #2 entry spacing
 {% don't indent paragraphs
  \setlength{\parindent}{0pt}
  % turn on hanging indent if param 1 is 1
  \ifodd #1
  \let\oldpar\par
  \def\par{\hangindent=\cslhangindent\oldpar}
  \fi
  % set entry spacing
  \setlength{\parskip}{#2\cslentryspacingunit}
 }%
 {}
\usepackage{calc}
\newcommand{\CSLBlock}[1]{#1\hfill\break}
\newcommand{\CSLLeftMargin}[1]{\parbox[t]{\csllabelwidth}{#1}}
\newcommand{\CSLRightInline}[1]{\parbox[t]{\linewidth - \csllabelwidth}{#1}\break}
\newcommand{\CSLIndent}[1]{\hspace{\cslhangindent}#1}
%% Extra package includes
\usepackage[nopostdot,nogroupskip,nonumberlist]{glossaries}
\usepackage{siunitx}

%% Glossary
%% Glossary

\newacronym{WEC}{WEC}{wave energy converter}
\newacronym{PM}{PM}{Pierson-Moskowitz}
\newacronym{PTO}{PTO}{power take-off}
\newacronym{KC}{KC}{Keulegan-Carpenter}
\newacronym{MPPT}{MPPT}{maximum power point tracking}



%% Fixes long table error with two columns
%% \makeatletter
%% \let\oldlt\longtable
%% \let\endoldlt\endlongtable
%% \def\longtable{\@ifnextchar[\longtable@i \longtable@ii}
%% \def\longtable@i[#1]{\begin{figure}[t]
%% \onecolumn
%% \begin{minipage}{0.5\textwidth}
%% \setlength{\LTcapwidth}{\linewidth}
%% \oldlt[#1]
%% }
%% \def\longtable@ii{\begin{figure}[t]
%% \onecolumn
%% \begin{minipage}{0.5\textwidth}
%% \oldlt
%% }
%% \def\endlongtable{\endoldlt
%% \end{minipage}
%% \twocolumn
%% \end{figure}}
%% \makeatother
\ifLuaTeX
  \usepackage{selnolig}  % disable illegal ligatures
\fi
\IfFileExists{bookmark.sty}{\usepackage{bookmark}}{\usepackage{hyperref}}
\IfFileExists{xurl.sty}{\usepackage{xurl}}{} % add URL line breaks if available
\urlstyle{same}
\hypersetup{
  pdftitle={Test author affiliations with lua filters},
  pdfauthor={Jane Doe,1,2,; John Q. Doe,2},
  hidelinks,
  pdfcreator={LaTeX via pandoc}}

\title{Test author affiliations with lua filters}
\author{Jane Doe\textsuperscript{$\dagger{}$,1,2,*} \and John Q. Doe\textsuperscript{$\dagger{}$,2}}
\date{\today}

\begin{document}
\maketitle
\begin{abstract}
Four Sentences: State the problem. Say why it is an interesting problem.
Say what your solution achieves, not what it is. Say what follows from
your solutions.
\end{abstract}

\textsuperscript{$\dagger{}$}
These authors contributed equally to this work.

\textsuperscript{1} Acme Corporation\\
\textsuperscript{2} Federation of Planets

\textsuperscript{*} Correspondence:
\href{mailto:jane.doe@example.com}{Jane Doe
\textless{}jane.doe@example.com\textgreater{}}

\hypertarget{introduction}{%
\section{Introduction}\label{introduction}}

\hypertarget{problem}{%
\subsection{Problem}\label{problem}}

\hypertarget{statement}{%
\subsubsection{Statement}\label{statement}}

\begin{quote}
What is the specific problem did you (attempt to) solve?\footnote{Getting
  to the point as soon as possible is very important. I do not recommend
  putting any background before stating what the point of your paper is.
  Sentence 1 is more important than the abstract.} Why exactly does this
problem exist? Who specifically has this problem? Scientists? Military?
Industry?
\end{quote}

\hypertarget{importance}{%
\subsubsection{Importance}\label{importance}}

\begin{quote}
How important is finding a solution (be realistic)? What would be
possible that previously was not? Who specifically would benefit and
how? Scientists? Military? Industry?
\end{quote}

\hypertarget{difficulty}{%
\subsubsection{Difficulty}\label{difficulty}}

\begin{quote}
How difficult is the problem to solve (be realistic)? What specifically
must be overcome? Are specific technologies required? Is there knowledge
required? Does it require organizational or institutional
support/cooperation?
\end{quote}

\hypertarget{solution}{%
\subsection{Solution}\label{solution}}

\hypertarget{proposed-solution}{%
\subsubsection{Proposed Solution}\label{proposed-solution}}

\begin{figure}
\hypertarget{fig:solution}{%
\centering
\includegraphics{solution.png}
\caption{Photo (or concept drawing) of your design preferably being
deployed in the field. This figure should appear on page one. Many
readers will not look at anything else. Caption is a short description
and summary of system functionality.}\label{fig:solution}
}
\end{figure}

\begin{quote}
Briefly what is your solution? Reference
fig.~\ref{fig:solution}.\footnote{Getting an impressive figure on page
  one is very important. Most people skim figures before they do
  anything else. Fig. 1 is more important than sentence 1.}
\end{quote}

\begin{quote}
What is different about your solution? How does it overcome the problem
difficulty?
\end{quote}

\hypertarget{design-considerations}{%
\subsubsection{Design Considerations}\label{design-considerations}}

\begin{quote}
What specific challenges does your solution face (briefly)? How are
those challenges different from those of other solutions?
\end{quote}

\hypertarget{success-criteria}{%
\subsubsection{Success Criteria}\label{success-criteria}}

\begin{quote}
How would you evaluate a design of your concept (be specific)? Why is
that criteria valid/important?
\end{quote}

\begin{quote}
Did you create a working prototype? Did it meet your sucess criteria?
Are further improvements likely?
\end{quote}

\hypertarget{background}{%
\section{Background}\label{background}}

\begin{quote}
I try to avoid including this section. It lengthens and defocusses the
paper. I try to push this information into references or appendix. I
assume my readers are either experts or know how to look things up. If
you need this, keep it as short as possible or attempt to fold it into
the Introduction and/or Related Work.
\end{quote}

\begin{quote}
Are there any conccepts required to understand the rest of your paper
that you would not already expect your readers to understand?\footnote{Look
  to reference texts books or comprehensive review papers.}
\end{quote}

\hypertarget{related-work}{%
\section{Related Work}\label{related-work}}

\begin{quote}
Keep this section as short as possible and try to combine with
Introduction and/or Background.
\end{quote}

\hypertarget{standard-practice}{%
\subsection{Standard Practice}\label{standard-practice}}

\begin{quote}
How is this (or similar) problem commonly or historically
solved?\footnote{Look to reference texts books or comprehensive review
  papers.} Why is this not good enough? How is this different than your
solution?
\end{quote}

\hypertarget{state-of-the-art}{%
\subsection{State-of-the-Art}\label{state-of-the-art}}

\begin{quote}
How is this (or similar) problem solved by entities with greater access
resources and the newest technology?\footnote{Look to reference texts
  books or comprehensive review papers.} Why is this not good enough?
Why is this not more common? How is this different than your solution?
\end{quote}

\hypertarget{key-projects}{%
\subsection{Key Projects}\label{key-projects}}

\begin{quote}
Are there any specific related projects you need to address? Why are
they not good enough? Why is this not more common? How is this different
than your solution?
\end{quote}

\hypertarget{system-overview}{%
\section{System Overview}\label{system-overview}}

\hypertarget{design-overview}{%
\subsection{Design Overview}\label{design-overview}}

\begin{quote}
What does the system look like? What are the functional subsystems? How
do the subsytems interact?
\end{quote}

\begin{quote}
For each subsystem: What is it? How does it work? How does it interact?
\end{quote}

\hypertarget{system-model}{%
\subsection{System Model}\label{system-model}}

\begin{quote}
I prefer to start with a broad picture and then drill down into details
rather than start with building blocks and build up into full structure.
This is an uncommon choice, but one that I believe strongly in.
\end{quote}

\begin{quote}
What is the overall state model? What are the state variables? How are
they related?
\end{quote}

\begin{quote}
What category of system or equation is this? (ie. harmonic oscillator)
What kinds of results does it predict?
\end{quote}

\hypertarget{system-elements}{%
\subsection{System Elements}\label{system-elements}}

\begin{quote}
For each term:
\end{quote}

\begin{quote}
What does this represent physically? How do you predict the term?
\end{quote}

\begin{quote}
What assumptions are you making? Are they realisitic, conservative,
conditional, or stupid?
\end{quote}

\begin{quote}
Can it be measured? Does measurement match prediction? Under what
conditions? What does this mean? Are only emperical measured values
valid?
\end{quote}

\hypertarget{system-dynamics}{%
\subsection{System Dynamics}\label{system-dynamics}}

\begin{quote}
Given characterisitics of each element, how is model expected to behave?
\end{quote}

\hypertarget{control}{%
\subsection{Control}\label{control}}

\begin{quote}
Is the system controllable?
\end{quote}

\begin{quote}
What are the control inputs?
\end{quote}

\begin{quote}
What are the control outputs?
\end{quote}

\begin{quote}
Is the system stable?
\end{quote}

\hypertarget{modeling-results}{%
\section{Modeling Results}\label{modeling-results}}

\begin{quote}
Why using models instead of models matched to lab results?\footnote{Model
  results can fold nicely into lab tests, but some projects rely heavily
  on modeling. This is fine, but it should be justified.}
\end{quote}

\begin{quote}
Are these model results reasonable?
\end{quote}

\begin{quote}
What assumptions are being made? Why is this valid?
\end{quote}

\hypertarget{lab-results}{%
\section{Lab Results}\label{lab-results}}

\begin{quote}
What needs testing?
\end{quote}

\begin{quote}
For each test: How did you set up the tests? How did you isolate
variables? How did you control inputs? How did you measure outputs? Did
results match expectations? What do results mean?
\end{quote}

\begin{quote}
Do lab results generally match expectations? Can a system be deployed
based on results?
\end{quote}

\hypertarget{field-results}{%
\section{Field Results}\label{field-results}}

\begin{quote}
Who? What? When? Where? How?
\end{quote}

\begin{quote}
Location, Depth, Duration, Distance?
\end{quote}

\begin{quote}
What are qualitative observations? Expectations vs Reality?
\end{quote}

\begin{quote}
What did you demonstrate?\footnote{This is a big deal and very
  impressive. While it only amounts to a paragraph or two of writing, it
  is the thing that makes a paper. Real artists ship, and real engineers
  deploy.}
\end{quote}

\begin{quote}
What were the major successes?
\end{quote}

\begin{quote}
What do the results mean?
\end{quote}

\hypertarget{conclusion}{%
\section{Conclusion}\label{conclusion}}

\hypertarget{success}{%
\subsection{Success}\label{success}}

\begin{quote}
Did you reach your success criteria? (be realistic)
\end{quote}

\begin{quote}
What does that mean?
\end{quote}

\hypertarget{conclusions}{%
\subsection{Conclusions}\label{conclusions}}

\begin{quote}
For each conclusion or major point: Restate the conclusion plainly.
\end{quote}

\hypertarget{future-work}{%
\subsection{Future Work}\label{future-work}}

\begin{quote}
What Technology Readiness Level (TRL)\footnote{I use the precise wording
  from TRL without actually refering to it. For example, TRL4
  -\textgreater{} ``validated in lab'', TRL6 -\textgreater{}
  ``demonstrated in the deep sea environment''} is the system in? What
is next for this project? What needs to be investigated further? What
needs to be tested?
\end{quote}

\hypertarget{acknowledgments}{%
\section{Acknowledgments}\label{acknowledgments}}

\begin{quote}
For each: This project was made possible by a grant from the Foundation
(123456)
\end{quote}

\begin{quote}
For each: We wish to acknowledge (person/organization) for (contribution
to project)
\end{quote}

\begin{quote}
For each: We wish to acknowledge the captains and crews of the R/V
Neverdock for their expertise and professionalism at sea.
\end{quote}

\hypertarget{references}{%
\section*{References}\label{references}}
\addcontentsline{toc}{section}{References}

\begin{quote}
Use multiple bib files (related.bib/background.bib/\ldots). Organize
within the files as well. Write your own abstracts into the entries.
\end{quote}

\begin{quote}
10-30 refs is reasonable. 15-25 makes more sense.
\end{quote}

\hypertarget{refs}{}
\begin{CSLReferences}{0}{0}
\end{CSLReferences}

\appendix

\hypertarget{appendix}{%
\subsection{Appendix}\label{appendix}}

\begin{itemize}
\tightlist
\item
  Background information required to understand new concepts.
\item
  Avoid using appendixes.
\end{itemize}

\end{document}
